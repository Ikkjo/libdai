\chapter{Change Log}
\hypertarget{changelog}{}\label{changelog}\index{Change Log@{Change Log}}

\begin{DoxyVerbInclude}
git HEAD
--------

* Fixed bug in include/dai/glc.h that confused swig [bug report by LaurentW]
* Fixed bug in example.py [bug report by Elise van der Pol]


libDAI-0.3.2 (2015-07-17)
-------------------------

Miscellaneous improvements:
* [Charles Vaske] Various improvements to EM code, including interface changes
* Added 'inclusive' argument to DAG::ancestors() and DAG::descendants()
* [Julien Rebetez] added FactorGraph::printDot() to SWIG interface
* Extended SWIG python interface (inspired by Kyle Ellrott): inference is possible in Python now!
* Added toString() formatting functions to the classes that had an output streaming operator<<
* Merged Generalized Loop Correction code kindly provided by Siamak Ravanbakhsh

Build system improvements:
* Build system: optional compilation targets are now set in include/dai/dai_config.h instead of Makefile.ALL

Bug fixes:
* Disabled debugging output in src/cobwebgraph.cpp to fix compilation issue under OSX
* [Lee C Baker] patch in include/dai/enum.h to fix compilation issues on OSX related to std::wistream and std::wostream (another workaround would be to specify -std=c++11)
* Fixed regression in SWIG interface caused by new build system (thanks to Benjamin Mumm for reporting this)
* Workaround for bug in Boost Graph Library version 1.54 ("The graph may not contain an edge with negative weight") by not using Prim's minimum spanning tree algorithm anymore
* Removed conversion script FastInf->libDAI because it is buggy (uses wrong permutation of factor entries)
* Replaced all occurrences of std::size_t by size_t to avoid SWIG problems
* Fixed division-by-zero issue in pow() which caused problems for GLC+ with sparse factors
* Fixed bugs in unit test prob_test.cpp: replaced all occurences of
  BOOST_CHECK_CLOSE(...,(Real)0.0,tol) with BOOST_CHECK_SMALL(...,tol)


libDAI-0.3.1 (2012-09-17)
-------------------------

Miscellaneous improvements:
* Full 64-bit support on all supported platforms (Linux, OSX, Windows)
* Added special MatLab interface for junction tree algorithm (dai_jtree)
* Optimized ClusterGraph( const FactorGraph&, bool) constructor

Build system improvements:
* Added Makefile.WINDOWS64, kindly provided by Lin Ming
* [Arman Aksoy] Added Makefile.MACOSX64
* Added VC10 build files, kindly provided by Sameh Khamis

Bug fixes:
* Fixed bug (found by Sameh Khamis): define NAN (Windows)
* Fixed bug (found by Sameh Khamis): add some necessary casts of size_t to BigInt (Windows)
* Fixed bug (found by Yan): replaced GNU extension __PRETTY_FUNCTION__ by __FUNCTION__ (Visual Studio) or __func__ (other compilers)
* Fixed bug (found by Andy Mueller): added GMP library invocations to swig Makefile
* Fixed bug (found by cax): when building MatLab MEX files, GMP/MPIR libraries were not linked
* Fixed bug in findMaximum (it only considered a single connected component of the factor graph)
* [Benjamin Piwowarski] Renamed "foreach" macro into "bforeach" to avoid conflicts with newer Boost library versions
* Fixed "typename" bug in src/alldai.cpp which resulted in FTBFS with older g++ compilers
* Fixed memory leak in alldai.cpp and removed the function builtinInfAlgs()
* Fixed memory leak in CBP::run()
* Fixed typo in Priya's name


libDAI-0.3.0 (2011-07-12)
-------------------------

New license:
* Changed license from GPL v2+ to FreeBSD (BSD 2-clause)

Miscellaneous improvements:
* Improved Exception class 
  (it now stores more information and doesn't print to stderr by default)
* Added builtinInfAlgs() and builtinInfAlgNames()

Bug fixes:
* [Laurens van der Maaten] src/matlab/dai.cpp now correctly handles missing logZ() 
  and maxDiff() implementations
* Now uses GMP big integer type to represent linear states and the number of
  states of VarSets (in some cases). This fixes a bug Jerome Maye and Priya 
  found in the State class. Unfortunately, the supplied Makefile.WINDOWS now 
  no longer works out-of-the-box.
* Fixed numerical issues in MF, FBP and TRWBP (discovered in sparse branch)
* Fixed bug in findMaximum(): inconsistent max-beliefs are now detected,
  instead of returning a MAP state with zero joint probability
  (reported by Hynek Urban)
* Fixed a Boost-related bug in src/util.cpp (reported by Avneesh Saluja)
  (the random seed needs to be an unsigned int on some platforms)
* Fixed two bugs in examples/example_sprinkler_gibbs (reported by Priya)
* Fixed bug in tokenizeString() and Evidence::addEvidenceTabFile()
  (a tab character at the end of a line was incorrectly discarded - bug reported by Priya)
* Replaced all Name members by name() virtual functions and removed global variable DAINames
  (this fixes a bug in matlab/dai.cpp reported by Thomas Mensink)


libDAI-0.2.7 (2010-08-19)
-------------------------

* Removed interfaces deprecated in 0.2.6.
* Fixed a bug in JTree::findMaximum() (reported by zhengyun84 and Dhruv Batra):
  if one or more variables had a MAP belief with more than one maximum, an 
  incorrect MAP state could result.


libDAI-0.2.6 (2010-08-05)
-------------------------

New features:
* Added decimation algorithm DecMAP for MAP inference (decmap.h/cpp).
* Added source code for one of the winning solvers of the 
  UAI 2010 Approximate Inference Challenge (examples/uai2010-aie-solver.cpp).
* Added support for reading files in the UAI Approximate Inference Challenge
  format, supporting both the 2006/2008 evidence file format and the 2010
  evidence file format (io.h/cpp).

Miscellaneous improvements:
* 'maxtime' property added to various algorithms (BP, Gibbs, HAK, TreeEP)
  for setting an upper limit on execution time.
* 'maxmem' property added to JTree for setting an upper limit on memory usage.
* SmallSet:
  - Added SmallSet::erase( const T& t ).
* VarSet:
  - nrStates() now returns a long double instead of a size_t.
* Permute:
  - Added ranges() accessor.
  - Added inverse() method.
  - Added optional reverse argument to Permute( const std::vector<Var>& ) constructor.
* Factor:
  - Added createFactorDelta( const VarSet& vs, size_t state ).
* ClusterGraph:
  - Added ClusterGraph(const FactorGraph& fg, bool onlyMaximal) constructor.
  - Added findCluster( const VarSet& cl ).
  - Added elimVar( size_t i ).
  - Added 'maxStates' argument to VarElim( EliminationChoice f, size_t maxStates=0 ).
* FactorGraph:
  - Added isMaximal(size_t).
  - Added maximalFactor(size_t).
  - Added FactorGraph::logScore( const std::vector<size_t>& statevec ).
* Gibbs:
  - Added 'restart' property.
  - Added Gibbs::findMaximum() for MAP inference.
* HAK:
  - Optimized region graph construction for HAK/GBP with clusters=BETHE.
* examples/examples_imagesegmentation.cpp:
  - [Peter Rockett] Improved Makefiles for image segmentation example.
  - Added build flag to specify the CImg version.
* util.h/cpp:
  - Added fromString<>( const std::string& x ).

Interface changes (the old interface is marked obsolete and will be removed in the next release):
* Added InfAlg::setMaxIter(size_t) for setting the maximum number of iterations.
* Deprecated Graph::Neighbor, BipartiteGraph::Neighbor and DAG::Neighbor
  and replaced them by a globally defined Neighbor structure
* Deprecated Graph::Neighbors, BipartiteGraph::Neighbors and DAG::Neighbors
  and replaced them by a globally defined Neighbors type
* Deprecated Graph::Edge, BipartiteGraph::Edge and DAG::Edge
  and replaced them by a globally defined Edge type
* Renamed 'iters' property of Gibbs algorithm into 'maxiter'.
* ClusterGraph::findVar( const Var& n ) no longer throws an exception if the
  variable is not found.
* Deprecated 'void tokenizeString( const string&, vector<string>&, const string& )'
  and replaced it with an improved version
  'vector<string> tokenizeString( const string&, bool, const string& )'
* Removed interfaces deprecated in 0.2.5.

Bug fixes:
* Fixed a problem with isnan() on FreeBSD; it is now in the dai namespace.
* Workaround for older g++ compilers (e.g. version 4.0.0 on Darwin 9.8.0)
  which have problems when comparing const_reverse_iterator with
  reverse_iterator.
* [Matt Dunham] Fixed a bug in Factors2mx in src/matlab/matlab.cpp
  (segfault if factors depending on 0 variables were passed to MatLab)


libDAI-0.2.5 (2010-05-09)
-------------------------

New features:
* Added unit tests for var, smallset, varset, graph, dag, bipgraph,
  weightedgraph, enum, util, exceptions, properties, index, prob, 
  factor, factorgraph, clustergraph, regiongraph, daialg, alldai
* [Ofer Meshi] Added a script for converting from FastInf to libDAI format
* Added utils/uai2fg, a utility to convert the UAI inference competition file
  formats to the libDAI factor graph file format
* Added dag.h/cpp, an implementation of Directed Acyclic Graphs
* Added examples/example_imagesegmentation

Miscellaneous improvements:
* TRWBP:
  - Added initialization of TRWBP weights by sampling spanning trees
* MF:
  - Added an "init" option that allows to initialize randomly
  - Added an "updates" option that allows to choose between standard
    mean field and "hard-spin" mean field
* MR:
  - rewrote response propagation implementation with help of BBP
  - now uses GraphAL to represent the Markov graph
  - miscellaneous smaller cleanups
* TreeEP:
  - Now also supports disconnected factor graphs
* JTree:
  - Now also supports disconnected factor graphs
* ExactInf:
  - Added findMaximum()
* RegionGraph:
  - Added error checking code
  - Changed the way a RegionGraph is streamed to an std::ostream.
* ClusterGraph:
  - Added nrVars() method
  - Added var( size_t ) method
  - Added cluster( size_t ) method
* FactorGraph:
  - clamped() now contains a delta factor for the clamped variable
  - Added MarkovGraph()
* TFactor:
  - Added takeAbs()
  - Added takeExp()
  - Added takeLog( bool )
  - Added operator-() const
  - Added sumAbs() const
  - Added operator==( const TFactor<T> &y )
  - Added get(size_t), set(size_t,T)
* TProb:
  - Added operator==( const TProb<T> &q )
  - accumulate() now also applies op2 to init
  - operator/= (T x) now yields 0 when dividing by 0
  - Added resize(size_t)
  - Added get(size_t) and set(size_t,T) 
  - Changed format of TProb<T> when streamed to an ostream
* multifor:
  - Added reset()
* PropertySet:
  - Added size()
  - Added clear()
  - Added erase()
* util.h/cpp:
  - Added std::string toString( const T& x )
* weightedgraph.h/cpp:
  - Added option to MaxSpanningTree and MinSpanningTree for
    choosing between Prim's algorithm and Kruskal's algorithm
  - Better error checking in RootedTree constructor
  - Added GraphEL::GraphEL( const GraphAL& G ) constructor
* BipartiteGraph:
  - Added hasEdge()
  - Added findNb1()
  - Added findNb2()
  - Added nb1Set()
  - Added nb2Set()
  - Added operator==( const BipartiteGraph& )
  - Added BipartiteGraph( size_t nr1, size_t nr2 ) constructor
  - Added operator<<( std::ostream&, const BipartiteGraph& )
* GraphAL:
  - Added hasEdge(size_t,size_t)
  - Added nbSet(size_t)
  - Added operator==( const GraphAL& )
  - Added operator<<( std::ostream&, const GraphAL& )
* SmallSet:
  - Added insert( const T& t ) method
* Improved regression test
* Made all "verbose" properties optional
* utils/createfg now has a new factor type ('ISINGUNIFORM') and the old
  'ISING' factor type has been renamed to 'ISINGGAUSS'
* tests/testdai option "marginals" now has five possible values: NONE
  outputs no marginals, VAR only variable marginals, FAC only factor
  marginals, VARFAC both types of marginals, and ALL outputs all
  marginals calculated by the algorithm.
* Compressed Makefile

Interface changes (the old interface is marked obsolete and will be removed in the next release):
* Removed RandomDRegularGraph(), please use createGraphRegular() instead
* TreeEP:
  - Changed construct( const RootedTree& ) into
    construct( const FactorGraph&, const RootedTree& )
* JTree:
  - Changed construct( const std::vector<VarSet>&, bool ) into
    construct( const FactorGraph&, const std::vector<VarSet>&, bool )
  - Changed GenerateJT( const std::vector<VarSet> & )
    into GenerateJT( const FactorGraph &, const std::vector<VarSet> & )
* RegionGraph: 
  - Made (previously public) members G, ORs, IRs and fac2OR protected and
    added DAG() accessor which returns a reference to the region graph DAG 
    structure (currently implemented as a BipartiteGraph).
  - Renamed calcCountingNumbers() into calcCVMCountingNumbers() and made it 
    protected, because exposing it to the world serves no purpose.
  - Removed partial constructor RegionGraph( const FactorGraph& fg )
  - Made RecomputeORs(), RecomputeORs( const VarSet& ) and RecomputeOR( size_t ) 
    protected and renamed them by changing the "Recompute" into "recompute", 
    because exposing them to the world serves no purpose.
* ClusterGraph:
  - Made (previously public) members G, vars and clusters private and added 
    bipGraph(), vars() and clusters() methods which offer read-only access 
    to these members.
  - Deprecated toVector()
  - Renamed size() into nrClusters()
* FactorGraph:
  - Renamed Cliques() into maximalFactorDomains() and made it faster
  - findVars( const VarSet& ) now returns a SmallSet<size_t>
    and its argument now has a const-qualifier (which fixes a small bug).
  - Made (previously public) member G private and added the 
    bipGraph() method, which offers read-only access to it.
  - Deprecated the iterator interface:
    o FactorGraph::iterator typedef
    o FactorGraph::const_iterator typedef
    o FactorGraph::begin() members
    o FactorGraph::end() members
  - Deprecated factor(size_t) which offered write access to a factor
    because using this functionality in the descendant RegionGraph is dangerous,
    as the outer regions are not automatically recomputed.
* TFactor:
  - Renamed states() into nrStates()
  - Added get(size_t) and set(size_t,T) operators; 
    get() is equivalent to "operator[](size_t) const" and set() should
    be used instead of the non-const operator[], which has been deprecated
* TProb:
  - Deprecated TProb( TIterator begin, TIterator end, size_t sizeHint=0 )
    constructor: the sizeHint argument no longer defaults to 0
  - Deprecated accumulate( T init, binOp op1, unOp op2 );
    instead, accumulateSum( T init, unOp op ) and
    accumulateMax( T init, unOp op, bool minimize ) should be used.
  - Deprecated TProb::NormType and TProb::DistType;
    ProbNormType and ProbDistType should be used instead.
  - Fixed bug by renaming operator<=() to operator<()
  - Added get(size_t) and set(size_t,T) operators; 
    get() is equivalent to "operator[](size_t) const" and set() should
    be used instead of the non-const operator[], which has been deprecated
* PropertySet:
  - Renamed Set() -> set()
  - Renamed Get() -> get()
  - Renamed GetAs<T>() -> getAs<T>()
  - Renamed ConvertTo<T>() -> convertTo<T>()
* util.h/cpp:
  - Removed max( const std::vector<Real> &v )
* weightedgraph.h/cpp:
  - Renamed MaxSpanningTreePrims into MaxSpanningTree
  - Renamed MinSpanningTreePrims into MinSpanningTree
  - The vertices in DEdge and UEdge are now named "first" and "second"
    for compatibility with a std::pair<size_t,size_t>
* BipartiteGraph:
  - delta1() and delta2() now return a SmallSet<size_t> instead of a vector<size_t>
* SmallSet:
  - The sizeHint argument of the iterator constructor
     SmallSet::SmallSet( TIterator begin, TIterator end, size_t sizeHint=0 )
    no longer has a default value in order to avoid confusion with the
     SmallSet::SmallSet( const T &t1, const T &t2 )
    constructor.

Bug fixes:
* [Stefano Pellegrini] Fixed bug in JTree::findMaximum()
* [Stefano Pellegrini] Fixed bug in BP[logdomain=1,inference=MAXPROD]
* Fixed some bugs in the MatLab interface build system
* Fixed a bug in utils/fginfo.cpp
* Fixed a bug in JTree::findMaximum() and simplified the code
* Fixed bug in BBPCostFunction::operator=() which prevented the desired 
  assignment from happening
* RegionGraph::WriteToFile( const char* ), RegionGraph::ReadFromFile( const char * )
  and RegionGraph::printDot( std::ostream& ) incorrectly called their respective
  FactorGraph ancestor methods; this has been corrected by letting them throw a
  NOT_IMPLEMENTED exception.
* Fixed bug in FactorGraph::clone()
* Fixed bug in min( const TFactor<T> &, const TFactor<T> & )
* Fixed bug in max( const TFactor<T> &, const TFactor<T> & )
* Fixed bug by renaming TProb<T>::operator<=() to TProb<T>::operator<()
* Fixed bug in PropertySet::setAsString<T>()
* Fixed a bug in rnd_seed()
* BipartiteGraph:
  - Fixed bug in eraseNode1()
  - Fixed bug in eraseNode2()
  - Fixed bug in isTree()
* GraphAL:
  - Fixed bug in nrEdges()
  - Fixed bug in addEdge()
  - Fixed bug in isTree()
  - Fixed bug in eraseNode()
  - Fixed bug in printDot()
* Fixed bug in createGraphGrid3D()
* Fixed bug in createGraphRegular()


libDAI-0.2.4 (2010-02-11)
-------------------------

New features:
* Implemented Tree-Reweighted BP
* [Frederik Eaton] Implemented Fractional Belief Propagation

Miscellaneous improvements:
* Improved documentation
* [Frederik Eaton] Added constructors for BBPCostFunction
* [Frederik Eaton] Added accessor/mutator for the elements of a SmallSet<>
* [Frederik Eaton] Added unary minus to TProb<>
* [Frederik Eaton] Added Hellinger distance to TProb<>::DistType
* [Frederik Eaton] Two small patches for src/jtree.cpp and utils/createfg.cpp
* tests/testdai option "marginals" now has three possible values: NONE
  outputs no marginals, VAR only variable marginals, and ALL outputs all
  marginals calculated by the algorithm.
* Strengthened convergence criteria of various algorithms
* Added GraphAL, an adjacency list implementation for graphs,
  similar to (but simpler than) BipartiteGraph
* Added some functionality to create various standard factors
  (the functions whose names start with "createFactor")
* Added examples example_sprinkler_gibbs and example_sprinkler_em
* Implemented various heuristics for choosing a variable elimination sequence 
  in JTree
* Added BETHE method for GBP/HAK cluster choice
* Made alias support official part of the library

Build system:
* Moved platform independent build options into Makefile.ALL
* Windows platform now needs Boost 1.37 or higher

Code cleanup:
* Cleaned up utils/createfg and tests/testdai
* Cleanup of some duplicate code in BP, FBP, TRWBP

Interface changes (the old interface is marked obsolete and will be removed in the next release):
* Removed obsolete/deprecated stuff
* Renamed Graph into GraphEL (for Graph, implemented as Edge List)
* Renamed some functions of BipartiteGraph:
    add1() -> addNode1()
    erase1() -> eraseNode1()
    nr1() -> nrNodes1()
    add2() -> addNode2()
    erase2() -> eraseNode2()
    nr2() -> nrNodes2()
* Renamed some functions in ClusterGraph:
    ClusterGraph::eliminationCost() -> ::eliminationCost_MinFill() 
    ClusterGraph::VarElim() -> ClusterGraph::VarElim( sequentialVariableElimination( ElimSeq ) ) 
    ClusterGraph::VarElim_MinFill() -> ClusterGraph::VarElim( greedyVariableElimination( eliminationCost_MinFill ) )
* Changed interface of one RegionGraph constructor
* Renamed "Clamped BP" into "Conditioned BP"

Bug fixes:
* Fixed regression in scripts/regenerate-properties (now all errors are 
  gathered before throwing the exception, as used to be the case)
* Fixed FTBFS errors on MACOSX (thanks to Sebastian Riedel for reporting this bug)
* [Alexander Schwing] Fixed error in BP that occured with damping
  and using log-domain updates
* Fixed bug in HAK for trivial region graphs (with only one outer region
  and no inner regions), reported by Alejandro Lage.
* Fixed long-standing bug in TreeEP (now, within-loop propagation optimization works)


libDAI-0.2.3 (2009-11-16)
-------------------------

New features:
* Complete doxygen documentation (although more detailed documentation still
  has to be written)
* [Charles Vaske] Added parameter learning for conditional probabilities by 
  Expectation Maximization
* [Patrick Pletscher] Added SWIG wrapper code for experimental python and 
  octave interfaces
* Added Max-Product functionality to JTree
* [Frederik Eaton] Added Back-Belief-Propagation
* [Frederik Eaton] Added approximate inference method "Clamped Belief Propagation"
* [Frederik Eaton] Added approximate inference method "Gibbs sampling"

Code cleanup:
* Updated copyright headers
* Cleaned up error handling by introducing the DAI_THROWE macro
* Introduced DAI_DEBASSERT macro to abbreviate a common pattern
* Replaced the standard assert() macro by DAI_ASSERT, which throws a
  dai::Exception and is even active if NDEBUG is defined
* Removed all the virtual default constructors *::create(), as they are not used
* [Frederik Eaton] Removed unnecessary copy constructors and assignment operators
* [Frederik Eaton] Change cout to cerr in warnings and error messages

Miscellaneous improvements:
* BipartiteGraph:
  - Added eraseEdge( size_t n1, size_t n2 )
  - [Frederik Eaton] Improved documentation of Neighbor
* TProb<T>:
  - Added divided_by(const TProb<T>&) const
  - Added constructor TProb<T>( const std::vector<S> &v )
  - Added size_t draw(), which draws a random index
  - [Frederik Eaton] Added setUniform()
  - [Frederik Eaton] Added sumAbs()
  - [Frederik Eaton] Added argmax()
  - [Giuseppe Passino] Added begin(), end()
* TFactor<T>:
  - [Giuseppe Passino] Added begin(), end()
  - [Charles Vaske] Added TFactor( const std::vector<Var> &vars, const std::vector<T> &p ),
    which reorders variables to canonical ordering before storing the table
  - Added constructor TFactor<T>( const VarSet &vars, const std::vector<S> &v )
  - Extended functionality of TFactor<T>::operators +,-,+=,-= to handle different VarSets
  - Added TFactor<T>::maxMarginal (similar to marginal but with max instead of sum)
* State:
  - [Giuseppe Passino] Added prefix ++ operator
* Permute:
  - [Charles Vaske] Added constructor in Permute for canonical variable ordering
* FactorGraph:
  - [Frederik Eaton] Added clampVar()
  - [Frederik Eaton] Added clampFactor()
  - [Giuseppe Passino] Added begin(), end()
* PropertySet:
  - Added keys(), for getting all keys
  - [Frederik Eaton] Added default constructor
  - [Frederik Eaton] Added constructor from string
* InfAlg:
  - Added setProperties(), getProperties() and printProperties()
  - [Frederik Eaton] Added beliefV(), beliefF()
* ExactInf:
  - Added calcMarginal(const VarSet &)
* HAK:
  - Added "init" parameter to allow for random initialization
* BP:
  - [Giuseppe Passino] Added findMaximum(), which can be used after running
    max-product BP to construct a global state with maximum probability
* JTree:
  - Added findMaximum(), which can be used after running
    max-product JTree to construct a global state with maximum probability
* MatLab interface:
  - MEX file dai now also returns variable and factor beliefs
  - [Sebastian Nowozin] MEX file dai now also optionally returns the MAP state
    (only for BP and JTree)
* README is now created automatically from doxygen documentation
* [Frederik Eaton] Added script for automatically generating properties code (used by CBP and BBP)
* [Frederik Eaton] Added newInfAlgFromString to alldai.h/cpp
* [Frederik Eaton] Added some utility macros (DAI_PV, DAI_DMSG, DAI_IFVERB)
  and functions (concat(),rnd()) to util.h
* Changed output stream operator<< for Var and VarSet
* [Frederik Eaton] Improved parsing code in tests/testdai to allow recursive
  alias substitutions
* [Frederik Eaton] Added useful aliases to aliases.conf
* [Giuseppe Passino] Optimised maximum-residual BP by using a reversed ordered
  set instead of the linear search (which can yield enormous speedups - a
  factor 500 has been measured on a binary Ising grid with 128x128 variables!)
* Added debug assertions to Var which check for inconsistent dimensions of
  variables with the same labels
* toc() now returns POSIX system time with maximum accuracy of microseconds
* Exception objects now remember their error code
* Examples:
  - Moved example.cpp to examples/
  - Added examples/example_bipgraph.cpp
  - Added examples/example_varset.cpp
  - Added examples/example_sprinkler.cpp
  - Added examples/example_permute.cpp
  - Added python and octave ports for examples/example_sprinkler.cpp

Build system:
* Improved Makefile:
  - Merged Makefile.win and Makefile.shared into Makefile
  - Macros DAI_VERSION and DAI_DATE have been introduced,
    in order to store this information in a central place
  - Introduced CCNODEBUGFLAGS in Makefile
  - Switched Makefile.win to GNU Make syntax
  - [Frederik Eaton] Added TAGS, lib targets to Makefile
* Compatibility:
  - [Peter Gober] libDAI can now be built on Cygwin
  - [Dan Preston] libDAI can now be built on Mac OS X
  - Changed regression test so that it also works under Windows

Interface changes (the old interface is marked obsolete and will be removed in the next release):
* Var:
  - Variable labels are now nonnegative (of type size_t)
* VarSet:
  - Replaced calcState() by non-member calcLinearState()
  - Replaced calcStates() by non-member calcState()
* Permute:
  - Renamed convert_linear_index into convertLinearIndex
* TProb<>:
  - Replaced log0() by log(true)
  - Replaced takeLog0() by takeLog(true)
  - Removed hasNonPositives()
  - Renamed minVal() to min()
  - Renamed maxVal() to max()
  - Renamed totalSum() to sum()
* TFactor<>:
  - Replaced log0() by TFactor<T>::log(true)
  - Removed divided_by, use operator/ instead
  - Removed divide, use operator/= instead
  - Removed partSum(const VarSet&), use marginal(const VarSet&,false) instead
  - Renamed minVal() to min()
  - Renamed maxVal() to max()
  - Renamed totalSum() to sum()
* FactorGraph:
  - Changed clamp and clamp interfaces (the variable to be
    clamped is now indicated by its index, not as a Var)
* RegionGraph:
  - Renamed Check_Counting_Numbers into checkCountingNumbers
  - Renamed Calc_Counting_Numbers into calcCountingNumbers
* Miscellaneous:
  - Renamed Treewidth(const FactorGraph &) into boundTreewidth(const FactorGraph &)
  - Replaced UEdgeVec by Graph
  - Replaced DEdgeVec by new RootedTree class
  - Moved functionality of GrowRootedTree() into constructor of RootedTree
  - Merged calcPairBeliefsNew() into calcPairBeliefs()
  - Removed calcMarginal2ndO()
  - Renamed smallSet<T> to SmallSet<T>
  - Removed class Diffs
  - [Frederik Eaton] Added backwards compatibility layer for edge interface to
    BipartiteGraph and FactorGraph (which will be obsoleted soon)

Bug fixes:
* Fixed reversal of min and max in 
  TFactor<T> min( const TFactor<T> &, const TFactor<T> & )
  TFactor<T> max( const TFactor<T> &, const TFactor<T> & )
* [Sebastian Nowozin] Fixed memory leak in MEX file dai
* [Frederik Eaton] In PropertySet::getStringAs(), check for typeid(ValueType) 
  before typeid(std::string) (this fixes a strange bug for empty string property)
* Added work-around for bug in Boost Graph Library
* Fixed bug in MaxSpanningTree (it wrongly assumed that the tree was not empty).
* Fixed bug in calcMarginal and calcPairBeliefs where states with zero probability 
  mass were clamped, leading to NaNs or assertion errors (thanks to Dan Preston 
  for reporting this)


libDAI-0.2.2 (2008-09-30)
-------------------------

New features:
* Approximate inference methods now report the number of iterations needed.
* Added damping to various algorithms to improve convergence properties.
* Added more features to utils/createfg for creating factor graphs.
* Added ExactInf class for brute force exact inference.
* [Giuseppe Pasino] Added "logdomain" property to BP, a boolean that controls
  whether calculations are done in the log-domain or in the linear domain;
  doing calculations in the log-domain may help if the numerical range
  of a double is too small.
* [Claudio Lima] Added Max-Product functionality to BP.
* Improved documentation.

Improved architecture:
* Added Exceptions framework.
* Pervasive change of BipartiteGraph implementation (based on an idea by
  Giuseppe Passino). BipartiteGraph no longer stores the node properties
  (former _V1 and _V2), nor does it store a dense adjacency matrix anymore,
  nor an edge list. Instead, it stores the graph structure as lists of
  neighboring nodes. This yields a significant memory/speed improvement for
  large factor graphs, and is more elegant as well. Iterating over neighbors is
  made easy by using boost::foreach.
* Added conditional compilation of inference methods.
* VarSet is now implemented using a std::vector<Var> instead of a
  std::set<Var>, which yields a significant speed improvement. Furthermore,
  the implementation has been generalized, resulting in the small_set<T> class
  which can be used to represent sets of small cardinality; VarSet is the
  specialization with T = Var.
* Improved ClusterGraph implementation, yielding significant speedups
  for the JunctionTree algorithm on large factorgraphs.

Code cleanup:
* Moved everything into namespace "dai".
* Renamed DEBUG to DAI_DEBUG to avoid conflicts.
* Replaced ENUM2,ENUM3,ENUM4,ENUM5,ENUM6 by single DAI_ENUM macro.
* Removed utils/remove_short_loops and matlab/remove_short_loops.
* Replaced sub_nb class in mr.h by boost::dynamic_bitset.
* Improved index.h:
  - Renamed Index -> IndexFor
  - Added some .reserve()'s to IndexFor methods which yields a
    25% speedup of testregression
  - Replaced multind by Permute
  - Added MultiFor
  - Added State
* New funcstionality of factor.h.
* Moved Properties and MaxDiff frameworks from InfAlg to each individual
  inference algorithm, because the Properties framework was not as
  convenient as I hoped, and not every inference algorithm needs a maxdiff
  variable. Also, replaced some FactorGraph functionality in InfAlg by a
  function that returns the FactorGraph. The result is cleaner (less
  entangled) code.
* Removed x2x.
* Replaced Complex with real numbers (negative potentials are just too rare
  to warrant the additional "complexity" :)).

Miscellaneous improvements:
* Now compiles also with MS Visual C++ (thanks to Jiuxiang Hu) and with
  GCC under Cygwin.
* Contributions by Peter Gober:
  - Renamed variable _N in mr.* for compatibility with g++ under Cygwin.
* Misc contributions by Giuseppe Passino:
  - removed "using namespace std;" from header files - bad practice;
  - moved header files in include/dai and sources in src;
  - changed #ifndefs to GNU style;
  - added extra warning checks (-W -Wextra) and fixed resulting warnings;
  - dai::TProb:
    o removed copy constructor and assignment operators (redundant);
    o implementation of some methods via STL algorithms;
    o added methods takeExp, takeLog, takeLog0 for transformation in-place;
    o explicit constructor (prevents implicit conversion from size_t to TProb);
    o added operator+,+=,-,-=, with argument T (for calculations in log-scale);
* Misc contributions by Christian Wojek:
  - New FactorGraph constructor that constructs from given ranges of factors
    and variables;
  - Optimization of FactorGraph constructors using tr1::unordered_map.
* FactorGraph constructors no longer check for short loops (huge speed
  increase for large factor graphs), nor for negative entries. Also, the
  normtype is now Prob::NORMPROB by default.
* Improved MaxSpanningTreePrims algorithm (uses boost::graph).

Interface changes:
* VarSet::
  - VarSet::stateSpace() -> nrStates(const VarSet &)
  - VarSet( const std::set<Var> ) -> VarSet( begin, end, sizeHint=0 )
  - VarSet( const std::vector<Var> ) -> VarSet( begin, end, sizeHint=0 )
  - removed bool operator||
  - operator&&(const VarSet&) -> intersects(const VarSet&)
  - operator&&(const Var&) -> contains(const Var&)
* FactorGraph::
  - delta(const Var &) -> delta(size_t)
  - Delta(const Var &) -> Delta(size_t)
  - makeCavity(const Var &) -> makeCavity(size_t)
  - vars() -> vars
  - factors() -> factors
  - removed MakeFactorCavity(size_t)
  - removed ExactMarginal(const VarSet &)
  - removed ExactlogZ()
  - removed updatedFactor(size_t)
  - removed _normtype and NormType()
  - removed hasShortLoops(...) and removeShortLoops(...)
  - WriteToDotFile(const char *filename) -> printDot( std::ostream& os )
  - undoProb(size_t) -> restoreFactor(size_t)
  - saveProb(size_t) -> backupFactor(size_t)
  - undoProbs(const VarSet &) -> restoreFactors(const VarSet &)
  - saveProbs(const VarSet &) -> backupFactors(const VarSet &)
  - ReadFromFile(const char*) returns void (throws on error)
  - WriteToFile(const char*) returns void (throws on error)
  - removed hasNegatives()
* RegionGraph::
  - nr_ORs() -> nrORs()
  - nr_IRs() -> nrIRs()
  - ORs() -> ORs
  - IRs() -> IRs
* *::Regenerate() -> *::construct()
* Renamed Index -> IndexFor
* Diffs:
  - max() -> maxDiff()
  - max_size() -> maxSize()
* Prob::max() -> Prob::maxVal()
* Factor::
  - max() -> maxVal()
  - part_sum() -> partSum()
* toc() in util.h now returns seconds as a double
* VarSet::operator&&
* Properties -> PropertySet


libDAI-0.2.1 (2008-05-26)
-------------------------

Bugfix release.
* added missing cstdio header in util.h
* fixed Properties in MR_CLAMPING_* and MR_EXACT_*
* added description of the factor graph fileformat
* improved Makefile


libDAI-0.2.0 (2006-11-30)
-------------------------

First public release.


0.1.5 (2006-11-30)
------------------

Regressions

- tests/testlcbp and tests/testlcbp are broken.
- EXACT method does not work anymore.
- The Properties framework gives a speed penalty because of the lookup
  costs involved; inner loops should be optimized.

General framework

- DAIAlg is now a template class; typedefs for DAIAlg<FactorGraph> and for
  DAIAlg<RegionGraph> are provided. In this way, we do not have to write "wrapper"
  functions to forward functionality from either FactorGraph or RegionGraph
  to DAIAlg. Functionality like clamping can be implemented in FactorGraph
  and in RegionGraph and no explicit interface is needed in descendants.
- New abstract base class InfAlg added, representing an inference algorithm,
  from which DAIAlg<T> inherits. This solves the incompatibility problems of
  DAIAlg<T> for different T (e.g. DAIAlg<FactorGraph> was incompatible with
  DAIAlg<RegionGraph>). More work is required to reduce code duplication
  (make FactorGraph part of InfAlg).
- Added generic interface (nrVars(), Vars(), nrFactors(), factor(size_t),
  beliefs(), belief(VarSet &), ...) to InfAlg and descendants.
- Added a saveProbs/undoProbs interface to InfAlg and descendants that enables
  one to save a few factors, modify them (e.g. clamp them), and then restore them
  to their old values. Undo should also init the corresponding messages / beliefs.
  This can be used if a given factor graph repeatedly needs to be clamped in
  different ways and an approximation method is run for each clamping; using the
  saveProbs/undoProbs can give a significant speed increase.
- Switched to a general Properties framework that handles the parameters of
  all inference methods in a uniform manner. The Properties class is a map of
  several properties in boost::any objects, indexed by their names (strings).
  It can read from a stream and write to a stream. It is recursive, in the sense
  that a Properties object can hold a variable of type Properties as well.
- Added a generic way of constructing inference algorithms from a factor graph,
  name and properties object. Added the newInfAlg function which constructs
  the requested object. This is used by LCBP, the Matlab interface and the
  command line (test) interface.
- Added a generic enum framework for enum parameters. Although implemented as a
  hack, it is quite useful in that it drastically simplifies and reduces the
  amount of code for handling enum parameters.
- Provided generic functions for calculating marginals in different ways that
  work for all approximate inference methods.

Bugfixes

- Fixed GBP free energy.
- Fixed bug in junctiontree (it didn't init the _vars variable).
- Corrected two bugs in operator&& and operator|| in VarSet (they returned
  the logical NOT of what they should return).
- Fixed bug in RegionGraph::RecomputeOR(s).
- Fixed bug in utils/create_dreg_fg:
  graph structure was not random for given parameters (forgot to call srand()).
- TreeEP bug workaround: use the complete junction tree instead of a subtree.
- Fixed bug in JTree::HUGIN() and JTree:ShaferShenoy() in case of junction tree
  that consists of one outer region only.
- Fixed INIT bug in LCBP2::UpdatePancake().
- Fixed MaxDiffs flow (except for MR).

New functionality

- HAK supports several default cluster choices:
  minimal (only factors)
  delta   (Markov blankets)
  loop    (all loops consisting of loops consisting of <loopdepth> or less variables)
  Only the maximal clusters are used as outer clusters.
- Implemented TreeEP. It generalizes the heuristic method described in the
  Minka & Qi paper for obtaining a tree with the most relevant interactions to
  higher order interactions. Almost all optimizations described in the Minka & Qi
  paper are used, except that evidence is passed over the whole tree instead of
  relevant subsets (the latter is almost implemented but buggy). Also added
  alternative (worst-case) algorithm that uses a maximum spanning tree on the
  weighted graph where the weight between neighbours i and j is given by
  N(psi,i,j), where psi is the product of all the factors involving both i and j
  (which is an upper bound on the effective interaction between i and j).
- Implemented MR (MontanariRizzo) based on Bastian's code, but extended it
  to be able to handle connectivities larger than 3 (in principle, up to 32).
  It supports different initialization methods (the original RESPPROP,
  the CLAMPING method and EXACT which uses JTree) and different update methods
  (FULL and LINEAR).
- Implemented LCBP2, an analogon of LCBP which represents pancakes as little
  networks and uses some approximate inference method on them for calculating
  marginals.
- Now there are several LCBP variants (LCBP, LCBPI, LCBPJ, LCBPK, LCBPL);
  LCBPJ works only for pairwise, LCBPK is LCBP improved for higher order
  interactions and LCBPL is LCBPI improved for higher-order interactions.
- Wrote one single program utils/createfg for creating various types of
  random factor graphs.
- Wrote utility to visualize factor graphs using graphviz.
  (it uses the BOOST Program Options library)
- Added fginfo utility that displays some info about a .fg file.
- Implemented Factor::strength function that calculates the potential strength
  N(psi,i,j) between variables i and j as described in cs.IT:0504030
- Wrote a general MatLab interface matlab/dai (similar to tests/test);
  this unified the matlab functions dai, dai_bp, dai_mf, dai_jt, dai_tep, dai_cvm.
- Added MATLAB routine that returns contraction matrix A for BP convergence analysis.
- Implemented a MATLAB interface ai_potstrength for Factor::strength
- Added Martijn's x2x

Improvements of existing code

- Reimplemented RegionGraph and descendants: a RegionGraph ISA FactorGraph
  and also a BipartiteGraph<FRegion,Region>. It now also keeps a map that
  associates outer region indices to factor indices (no powers yet, this
  is deemed superfluous) and provides functions to recompute (some of) the
  outer regions from the factors.
- InfAlg descendants run() methods now stop immediately and return NAN in case
  they detect NANs. Only BP does not do NAN checking for performance reasons.
- LCBP now works with factors containing zeroes (by defining x/0 = 0).
- HAK, GBP and DoubleLoop now conform to the standards for verbose reporting,
  timing and convergence criteria.
- Implemented logZ() for JTree. It does the calculation during message-passing.
- Marginal2ndO now optionally divides by the single node beliefs (to the power n-2);
  hopefully this will give more accurate approximations.
- Marginal and Marginal2ndO (optionally) use the new saveProbs/undoProbs functionality
  for a faster way of calculating marginals, which does not require a call to init()
  nor cloning the whole object for each clamping. This leads to a significant speedup.
- LCBP (and LCBP2) now have complete flexibility in the specification of the
  inner method, i.e. the method used to generate the initial cavity approximations.
  One can pass two strings, a name and a properties string, and LCBP simply invokes
  newInfAlg to construct the corresponding inference algorithm and uses the generic
  marginal functions to approximate cavity marginals.
- Replaced the global "method" variable by local properties and removed ai.h
- Added some methods to Factor (operators *, *=, /, /= with doubles as
  second argument, operators -, +=, -= with other Factors as second
  arguments, randomize(), RemoveFirstOrderInteractions) and similar
  operations to Prob
- Moving towards boost::program_options for handling command line arguments
  (tests/test is done).
- Renamed some FactorGraph methods:
  nr_vars       -> nrVars
  nr_factors    -> nrFactors
  varind        -> findVar
  factorind     -> findFactor
  makeFacCavity -> makeFactorCavity
- LCBP_SEQMAXRES has been removed because it did strange things.
- Implemented RandomDRegularGraph
- Implemented JTree::calcMarginal for marginals that are not confined
  within one cluster (using cut-set conditioning).
- Added isConnected() method to FactorGraph (some methods do not work with
  disconnected factor graphs).
- Pair beliefs are now calculated in a symmetrical way by calcPairBeliefs
- Removed single node interaction "correction" code from clamping methods
- Removed calcCavityDist and calcCavityDist2ndO
- No longer depends on GSL.
- Increased portability by combining platform dependant utility functions
  in util.{h,cpp}.
- Wrote *.m files providing help

Testing framework

- Made a new and significantly improved testing framework that provides most
  functionality from the command line.
- The basis is provided by tests/test, which uses the newInfAlg functionality
  and enables the user to easily compare from the command line different
  inference methods on a given factorgraph. All parameters can be specified.
  Output consists of CPU time, average and maximum single variable marginal
  errors, relative logZ error and MaxDiff().
- tests/aliases.conf contains aliases for standard combinations of methods
  and their options (which can be used in tests/test).
- tests/large contains several bash/python scripts that create random factor
  graphs, compare several approximate inference algorithms (using tests/test) and
  allow for easy visualization of the results using PyX.
- Added several .fg files for test purposes to /tests (e.g. two ALARM versions
  alarm.fg and alarm_bnt.fg; testfast.fg, a 4x4 periodic Ising grid for
  regression testing).
- Added a regression test to the Makefile which is included in the standard
  target.  It compares all inference methods on tests/testfast.fg with the
  results stored in tests/testfast.out

Miscellaneous

- Expanded all tabs to spaces (":set tabstop 4\n:set expandtab\n:retab" in vim)
- Experimental MATLAB code added for StarEP approximation on cavity
- Renamed project to libDAI and changed directory name accordingly.
- Renamed JunctionTree to JTree.
- Fixed licensing (now it's officially GPL).
- Improved README


revision 252
------------

Functionality
- Added RegionGraph, GBP, CVM and HAK (double-loop).
- Added JunctionTree (with two update algorithms, HUGIN and Shafer-Shenoy), which is a
  RegionGraph.
- NormType is now chosen automatically (in case of negative factors, Prob::NORMLINF is used,
  otherwise the default Prob::NORMPROB is used). Also, in case of negative factors, the
  RegionGraph constructors assign each Factor to a unique outer region instead of dividing
  it over all subsuming outer regions. See README for when negative factors are known to work
  and when not.
- FactorGraph::FactorGraph(const vector<Factor>) only gives a warning in case of short loops,
  it does not automatically merge factors anymore.
- Removed BP_SEQMAXRESNOCLEAR (all cavity initialization methods now are implicitly NOCLEAR)
- Added MATLAB interface functions ai_readfg, ai_removeshortloops and ai_bp
- Added LCBP-III type that should be equivalent to LCBP-II, but can handle zeroes
  in potentials. Note that it is significantly slower than LCBP-II (and has to be reimplemented
  such that it does not store the complete pancakes, but represents them as little factor graphs).

Implementation / code
- Made a seperate type WeightedGraph, which until now only implements Prim's
  maximal spanning tree algorithm and is only used by the junction tree code. It might
  be good to make it a class somewhere in the future and extend it's interface.
- Made a seperate class ClusterGraph, which is only used by the junction tree
  code. It's main purpose is a graph-theoretical variable elimination algorithm.
- Implemented the heuristic "minimum-new-edges-in-clique-graph" for variable elimination.
- Massive code cleanup, moving towards "generic" programming style, using
  multiple inheritance and polymorphism.
  o BP, LCBP, MF, HAK and JunctionTree now inherit from a common DAIAlg class
  o Made generic functions Marginal, Marginal2ndO, calcCavityDist, calcCavityDist2ndO, clamp
    that can be used by FactorGraph-based DAIAlgs.
  o Created TProb<T> class, which stores a probability vector (without the accompanying indexing
    and VarSet) and provides functionality for it (which is used by TFactor<T>).
  o Rewrote the VarSet class. It now caches its statespace(). It now privately inherits from set<Var>.
    I had to overload the insert methods of set<Var> so that they calculate the new statespace.
  o Rewrote the TFactor class. The TFactor class now HAS a TProb and HAS a VarSet.
- Rewrote BP to use the new TProb<T> interface. Performance of BP improved up to a factor 6 by:
  o using Prob's instead of Factor's;
  o splitting the multiplication of the messages into two parts (thanks to Vicenc!);
  o optimizing the calculation of the beliefs (only the message calculations were optimized till now).
  o replacing FactorGraph::_nb1 and _nb2 (which were set<size_t>) by vector<size_t>
- LCBP now seperately stores cavitydists and pancakes. Added InitPancakes() method
  that takes the cavitydists and multiplies them with the relevant factors. This
  resulted in an API change in AI which now accepts and returns initial cavitydists
  instead of initial pancakes.

Minor changes
- Started writing DoxyGen documentation
- Renamed lcptab2fg.m matlab/ai_writefg.m
- Moved all matlab stuff to matlab/
- More detailed reporting (also reports clocks used).
- Marginal and Marginal2ndO now use *differences* in logZ to avoid NaNs.
- Improved testing suite.
- Removed logreal support.
- FactorGraph now also supports input streams and ignores comment lines in .fg files.
- Added tests/create_full_fg.cpp and tests/create_ising_fg.cpp which create
  full and periodic 2D Ising networks according to some command line parameters.
- Now logZ really returns logZ instead of -logZ.
- Added FactorGraph::WriteToDotFile


0.1.4 (2006-04-13)
------------------
- Added file IO routines to read and write factorgraphs.
- Added L-infinity normalization for use with (partially) negative factors.
- Renamed BetheF, MFF to logZ, which now use complex numbers to be able to
handle (partially) negative factors.
- Added test suite.
- All probabilities are now represented using double instead of LogReal<double>.
- Combined Alg and Alg3 into LCBP. Several update schemes possible.
- Combined several variants of BP into doBP. Several update schemes possible.
Now uses precalculated indices for optimization.
- Renamed Node -> Var and Potential -> Factor.
- Extensive code cleanup. More use of OO techniques. Changed API.
- MaxIter now means maximum number of passes (corresponding to number of
_parallel_ updates).
- MaxDiff now means the maximum L-infinity distance between the updated and
original single variable beliefs, for all AI methods.
- Implemented RBP which converges faster than BP for difficult problems.
- Now uses method parameter which is a bitmask combining outmethod and inmethod
(see ai.h).


0.1.3   (2006-03-22)
--------------------
- All AI methods now return maxdiff
- ai.cpp:
    o Now returns maxdiffinner and maxdiffouter
    o New BP2ndO innermethod (estimate only 2nd order cavity interactions)
    o New InitCav outermethod (only create initial cavity distributions)
- bp.cpp:
    o New CavityDist2ndO which estimates 2nd order cavity interactions
- Makefile:
    o Bugfix: removed dependencies on algwim.*


0.1.2   (2006-02-28)
--------------------
- Cleaned up alg.cpp (removed Alg2 and its corresponding data structures).
- Added the possibility to provide initial cavity distributions as an input
argument to ai (not much error checking is done, so be careful).
- Potentials2mx now correctly sets the dimensions of the P field (i.e. for
the output arguments Q, Q0 of ai).
- Removed algwim.* since it does not work.


0.1.1   (2006-02-28)
--------------------
- The constructors of (Log)FactorGraph and LogFactorGraph from a
vector<(Log)Potential> now merge potentials to prevent short loops (of length
4) in the factor graph. These are used in ai to construct the factor graphs
from the psi argument. If compiled with DEBUG defined, the method calc_nb()
of BipGraph checks for the existence of short loops.
- Changed calling syntax of ai (now the actual syntax *does* correspond to its
description in the help).
- ai does not hook cout anymore (which caused weird segfaults).
- Fixed a bug in an assert statement in the matlab interface code in ai.cpp.
- Removed network.* since it is not useful.


0.1.0   (2006-02-28)
--------------------
First version worthy of a version number.
\end{DoxyVerbInclude}
 